%\documentclass[a4paper,12pt]{article}
\documentclass[a4paper,12pt]{scrartcl}
\usepackage[usenames,dvipsnames]{color}

\usepackage{pgfplots, pgfplotstable}
\usepackage{fancyhdr}
\usepackage[utf8]{inputenc}				% encodage UTF-8
\usepackage[T1]{fontenc}
%\usepackage{fullpage}					% supprime marges
\usepackage[francais]{babel}
\usepackage{graphicx}
\usepackage{amsmath}
\usepackage{amssymb}
\usepackage{sectsty}
\usepackage{enumerate}
\usepackage{float}
\usepackage{listings}
\usepackage{listingsutf8}				% package PAS PAR DEFAUT + ? \usepackage{textcomp}


\usepackage[colorlinks=true, urlcolor=MidnightBlue, linkcolor=Black]{hyperref}

\lstset{
	tabsize=4,
	frame=single,
	breaklines=true,
	basicstyle=\ttfamily\small,
	frameround={tttt},
	showstringspaces=false,
	language=C,
	keywordstyle=\color{blue},
	commentstyle=\color{OliveGreen},
	stringstyle=\color{red}\itshape,
	inputencoding=utf8/latin1,
}

\pagestyle{fancy}
\setkomafont{disposition}{\normalfont\bfseries}

\lfoot{}
\cfoot{Détails sur l'évaluation du dossier de candidature}
\rfoot{\thepage}
\renewcommand{\headrulewidth}{0.0pt}
\renewcommand{\footrulewidth}{0.4pt}
%\renewcommand{\chaptermark}[1]{ \markboth{#1}{} }
\renewcommand{\sectionmark}[1]{ \markright{#1}{} }

%\fancyhead[R]{\textit{\nouppercase{\rightmark}} }
\fancyhead[R]{}
\fancyhead[CL]{}
%\fancyhead[LO]{\textit{\nouppercase{\rightmark}} }

%\textwidth 16cm
\textheight 24cm
\hoffset 0mm
\voffset 0.5cm

\sectionfont{
    \sectionrule{0pt}{0pt}{-5pt}{0.8pt}
}

\title{\vfill Détails sur l'évaluation du dossier de candidature}
\subtitle{Championnats de France de Rubik's Cube 2014}
\author{Gaël Dusser, Mario Laurent, Hippolyte Moreau\\Philippe Virouleau, Thomas Watiotienne}
\date{7 octobre 2014\vfill}


%===============
\begin{document}
%===============




\maketitle



\pagebreak
%~
%\newline

%---------
%	Introduction
%---------


Ce document décrit les attentes de la commission concernant le dossier de candidature pour l’organisation des Championnats de France de Rubik’s Cube. Il contient certains impératifs, des indications sur les points importants que la commission étudiera, mais aussi des informations et conseils variés, basés sur notre expérience et les éditions précédentes.

Il n’est pas attendu que tous les points soient respectés à la perfection, mais que le dossier soit, d’une manière générale, digne d’une candidature pour une compétition de cette importance.




\section*{Préambule : Winning Moves}

Winning Moves a été initiateur du Championnat de France depuis 2003. En 2013, l’entreprise a arrêté de proposer le championnat, et a donc laissé la communauté s’en occuper. 
Pour cette première édition organisée par la communauté, Winning Moves a été le financeur principal, mais cette option n’est pas obligatoire, et entraine certaines contraintes : pas de sponsor jeu non-affilié Winning Moves, par exemple.




\section*{Équipe d’organisation}


Nous prendrons en compte la taille et l’expérience de l’équipe d’organisation, il faudra être le plus explicite possible sur les rôles de chacun au sein de l’équipe. Afin d’éviter les quiproquos, merci d’indiquer les id WCA des membres.




\section*{Date}

Pour des raisons historiques, ainsi que pour faciliter l’envoi des gagnants au prochain championnat majeur le cas échéant, nous souhaitons que les Championnats de France se déroulent un week end entre fin mars et fin avril 2014, dans la mesure du possible.



\pagebreak

\section*{Lieu}


\subsection*{La ville}

L’accessibilité de la ville accueillante est un critère primordial. Les compétiteurs arrivent de la France entière, éventuellement de l’étranger, et doivent pouvoir se rendre dans la ville par les transports en commun, notamment.


\subsection*{La salle}
Elle doit être suffisamment grande pour accueillir tous les compétiteurs confortablement. En fonction de la ville hôte, l’affluence peut varier de 100 à 200 personnes. 
Le règlement WCA impose une zone de mélange séparée de la zone de compétition, ainsi qu’une zone d’attente, elle aussi séparée, afin de réduire les risques de triche.
La salle devra également être suffisamment éclairée, en particulier la zone de compétition, et la température devra être adéquate pour permettre à tout le monde de concourir dans les meilleures conditions.


\section*{Site Internet}


Le site Internet de la compétition devra proposer une version anglaise, afin que les étrangers souhaitant participer puissent avoir toutes les informations. Dans la mesure du possible il devra être clair et complet. Des membres de la commission peuvent vous aider sur ce point.




\section*{Épreuves}


Hormis la présence de l’intégralité des épreuves officielles, aucun impératif n’est imposé. Cependant, la commission se réserve le droit de proposer des modifications sur le nombre de tours par épreuve ainsi que les time limits envisagées




\section*{Planning}


Le planning définitif n’est pas demandé, mais une ébauche complète est indispensable : épreuves prévues, time limits et cutoffs éventuels pour chaque épreuve, nombre de tours et de qualifiés pour chaque épreuve, répartition du tout sur la durée de la compétition. L’épreuve reine étant le Rubik’s Cube 3x3x3, il est vivement conseillé d’en proposer 3 tours.




\section*{Budget}


Le budget dépendra principalement de vos sponsors et subventions, vous êtes libre de sa répartition, en prenant en compte les conseils des sections suivantes.


\section*{Sponsors}


Vous êtes libre du choix de vos sponsors, pensez simplement à nous les indiquer. Rappellons tout de même que certains sponsors refusent d’être associés à d’autres et qu’il faut donc être vigilant quant à leur choix.


\section*{Prix et récompenses}


Plusieurs types de prix et récompenses sont possibles. Les éditions précédentes ont proposé, notamment, des prix sous forme d’argent, des lots pour les podiums, des invitations aux championnats internationaux, des remboursements du trajet au CdF pour les subX ou les N premiers...
Afin de rendre les Championnats de France attractifs et accessibles pour le plus grand nombre, il sera apprécié qu’une partie du budget “lots” soit consacrée au remboursement des trajets, dans les limites de votre choix.


\section*{Dispositions supplémentaires}


Aucune disposition supplémentaire n’est requise, cependant un partenariat pour le logement, le transport ou la restauration par exemple, sera un plus.
Bien entendu, la réalisation d’idées nouvelles, originales, ou attendues par la communauté, sera probablement appréciée !









%=============	
\end{document}
%=============
