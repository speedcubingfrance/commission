\documentclass[a4paper,12pt]{scrartcl}
\usepackage[usenames,dvipsnames]{color}

\usepackage{pdfpages}
\usepackage{pgfplots}
\usepackage{fancyhdr}
\usepackage[utf8]{inputenc}				% encodage UTF-8
\usepackage[T1]{fontenc}
\usepackage[francais]{babel}
\usepackage{sectsty}


\usepackage[colorlinks=true, urlcolor=MidnightBlue, linkcolor=Black]{hyperref}

\pagestyle{fancy}
\setkomafont{disposition}{\normalfont\bfseries}

\lfoot{}
\cfoot{Commission Championnats de France 2019}
\rfoot{\thepage}
\renewcommand{\headrulewidth}{0.0pt}
\renewcommand{\footrulewidth}{0.4pt}
\renewcommand{\sectionmark}[1]{ \markright{#1}{} }

\fancyhead[R]{}
\fancyhead[CL]{}
\textheight 24cm
\hoffset 0mm
\voffset 0.5cm

\sectionfont{
    \sectionrule{0pt}{0pt}{-5pt}{0.8pt}
}



%===============
\begin{document}
%===============
\sloppy

\begin{titlepage}
    \centering
    \vfill
    \includegraphics[width=\textwidth]{logoafsletters.png}
    \vfill
    {\bfseries\Huge
        Critères de décision pour l'attribution de l'organisation des Championnats de France 2019\\
		\vskip1cm
	\large
        Document destiné aux équipes d'organisation candidates\\
        \vskip3cm
        Pauline Bonnaudet, Emilien Fabre, Arthur Garcin, Hippolyte Moreau\\
        \vskip2cm
\today
    }    
    \vfill
\end{titlepage}


\pagebreak
%~
%\newline

%---------
%	Introduction
%---------

\section*{Préambule}

Ce document est destiné aux équipes candidates à l'organisation des championnats de France
de speedcubing 2019.

La première partie présente les modalités de candidature et le contenu attendu de celle-ci.

La seconde donne une idée des critères de sélection que la Commission va utiliser
pour départager les candidats. Elle décrit les attentes de la Commission concernant
le dossier déposé. Elle contient également des conseils sur les points importants des candidatures.


\section*{Charte de la Commission}
%% TODO ajouter lien%%
La Commission fixe ses objectifs et effectue son travail en s'appuyant sur la Charte de l'AFS pour la Commission pour l'attribution des Championnats de France (voir Annexe).
 
Cette Charte définit les rôles et devoirs des équipes candidates, de la Commission et de l'équipe organisatrice. En envoyant un dossier de candidature, vous reconnaissez avoir lu la Charte susnommée et en accepter le contenu. Tout dossier qui mènerait à un non-respect de cette Charte sera directement écarté.


\section*{Dépôt des candidatures}

La date limite de dépôt des candidatures est fixée au 1er mars 2018 à 23h59. Le dépôt se fait auprès de la Commission des championnats de France, à l'adresse \href{mailto:cdf-2019@speedcubingfrance.org}{cdf-2019@speedcubingfrance.org}.
Le format privilégié est le pdf ; si votre dossier contient d'autres détails qui ne seraient pas communicables au format pdf, merci de réaliser une archive (zip ou tar) contenant tout ce que vous souhaitez nous faire parvenir. Aucun dossier comprenant un membre de la Commission dans son équipe d'organisation ne sera accepté.

Le date a été fixée dans l'optique de laisser un temps de réflexion et de dialogue avec les équipes ayant déposé un dossier, suivi d'une annonce officielle du lieu et de la date des Championnats de France 2019 lors des Championnats de France 2018 (Lyon, 31 mars - 2 avril). Si vous souhaitez être candidats à l'organisation mais rencontrez un souci lié au calendrier décrit ci-dessus, nous vous invitons à contacter la Commission au plus vite afin d'en discuter ensemble.

Comme évoqué ci-dessus, la commission est susceptible de vous demander des informations complémentaires afin de faire son choix. Le moyen de communication privilégié sera un appel vidéo entre la commission et l'équipe d'organisation. Soyez donc prêts à répondre favorablement à une requête comme celle-ci entre le dépôt de votre dossier et le 25 mars.

Dans les jours suivants la date de remise des dossiers, la Commission annoncera publiquement le nombre de dossiers reçus ainsi que les villes candidates.

\pagebreak
\section*{Évaluation des candidatures}

Cette partie décrit les attentes de la Commission concernant les dossiers de candidature. Il contient certains impératifs, des indications sur les points importants que la Commission étudiera, mais aussi des informations et conseils variés, basés sur notre expérience et les éditions précédentes. Il n'est pas attendu que tous les points soient respectés à la perfection, mais que le dossier soit, d'une manière générale, digne d'une candidature pour une compétition de cette importance.


\subsection*{Équipe d'organisation}

La Commission prendra en compte la taille et l'expérience de l'équipe d'organisation, il faudra être le plus explicite possible sur les rôles de chacun au sein de l'équipe. Afin d'éviter les quiproquos, merci d'indiquer les id WCA des membres de l'équipe organisatrice ayant déjà participé à des compétitions. Également, dans le but de pouvoir vous contacter facilement si nécessaire lors de l'évaluation des dossiers, nous vous demandons de nous fournir une adresse email générale pour l'équipe ainsi que les numéros de téléphone et adresses emails personnelles d'au moins 2 membres principaux de l'équipe d'organisation.


\subsection*{Date}

Pour des raisons historiques, ainsi que pour faciliter le \emph{sponsoring} des gagnants au prochain championnat majeur le cas échéant, il reste préférable, dans la mesure du possible, que les Championnats de France se déroulent un week end entre fin mars et début mai 2019.
Les week-ends de trois jours sont à privilégier en raison du nombre probablement très élevé de compétiteurs intéressés. Une compétition sur deux jours risque de rapidement devenir très compliquée à organiser sans accepter de nombreux sacrifices (peu de tours par épreuves, cutoffs restrictives, nombre de compétiteurs trop limité, etc.). Bien que cela ne soit pas impossible, soyez conscients que l'organisation d'un Championnat sur deux jours sera particulièrement contraignante.

\subsection*{Lieu}

L'accessibilité de la ville accueillante est un critère primordial. Les compétiteurs arrivent de la France entière, éventuellement de l'étranger, et doivent pouvoir se rendre dans la ville par les transports en commun, notamment.

La salle doit être suffisamment grande pour accueillir tous les compétiteurs confortablement.
Si l'affluence peut dépendre de la date et de la ville hôte, nous rappelons que la communauté française de speedcubing est en pleine expansion. Pour cette raison, les dossiers proposant des salles pouvant accueillir 200 à 300 compétiteurs et leurs accompagnants seront valorisés.

\href{https://www.worldcubeassociation.org/regulations/translations/french/#article-7-environment}{Le règlement de la WCA impose de nombreuses contraintes sur le lieu d'accueil}.
Entre autres une zone de mélange séparée de la zone de compétition, ainsi qu'une zone d'attente, elle aussi séparée, sont requises afin de réduire les risques de triche.
La salle devra également être correctement éclairée (lumière blanche, éclairage naturel si possible en journée), en particulier la zone de compétition, et la température devra être adéquate pour permettre à tout le monde de concourir dans les meilleures conditions.


\subsection*{Site Internet}

Le site Internet de la compétition devra proposer une version anglaise, afin que les étrangers souhaitant participer puissent avoir toutes les informations.
Dans la mesure du possible il devra être clair et complet. Des membres de la Commission peuvent vous aider sur ce point, et si besoin le site internet peut être hébergé par l'AFS. À noter qu'il est aussi possible d'utiliser les fonctionnalités du site internet de la WCA qui propose à présent une alternative pratique aux sites internet externes (possibilité de créer l'équivalent d'un site internet, avec plusieurs onglets et possibilités d'inclure des liens, des images, etc.).


\subsection*{Planning}

Hormis la présence de l'intégralité des épreuves officielles, aucun impératif n'est donné. Cependant, la Commission se réserve le droit de proposer des modifications sur le nombre de tours par épreuve ainsi que les time limits envisagées.

Le planning définitif n'est pas demandé, mais une ébauche complète est indispensable : épreuves prévues, \emph{time limits} et \emph{cutoffs} éventuels pour chaque épreuve, nombre de tours et de qualifiés pour chaque épreuve, répartition du tout sur la durée de la compétition. L'épreuve reine étant le 3x3x3, il est vivement conseillé d'en proposer 3 ou 4 tours et sans modification de la \emph{time limit} par défaut (10 minutes), afin qu'un maximum de personnes puisse y prendre part.

Si la compétition est organisée sur trois jours, prenez en compte dans votre réflexion qu'il peut être judicieux de programmer les épreuves les moins populaires le vendredi.

\subsection*{Ouverture}
Depuis maintenant plusieurs années, toutes les compétitions doivent être ouvertes aux compétiteurs de toutes nationalités, sauf accord explicite de la WCA. La communauté étant désormais habituée à ce fonctionnement qui nous semble respecter l'esprit de partage de la discipline, la Commission demande donc que les Championnats de France restent ouverts à tous.


\subsection*{Budget}

Le budget dépendra principalement de vos sponsors et subventions, vous êtes libre de sa répartition.

La Commission peut vous donner les pistes suivantes :

\begin{itemize}
    \item Démarcher les conseils généraux, les instances locales, les comités départementaux olympiques et sportifs.
    \item Démarcher les entreprises locales en leur faisant par la suite un peu de publicité peut être une solution très intéressante.
    \item Démarcher les magasins de jeux et les principaux sites revendeurs de speedcubes qui peuvent accepter d'envoyer des lots.
\end{itemize}

Ces points ne sont que des pistes de réflexion pour vous permettre de mettre en place votre dossier, vous restez libre de choisir les sponsors que vous souhaitez. 
Attention cependant à ce que peuvent vous demander ces sponsors en échange de leur participation. Dans le cas de grandes entreprises ou institutions, il sera préférable d'établir un contrat. Assurez-vous également que les rôles entre l'équipe d'organisation et les sponsors soient bien définis, afin que vous restiez maîtres de l'événement.

Le budget général nous permettant de juger de la faisabilité de l'ensemble de vos propositions, nous vous conseillons d'accorder un soin tout particulier à son élaboration.


\subsection*{Gratuité}
Bien que la plupart des pays fassent payer l'inscription aux compétitions, ce n'est que très rarement le cas en France. Jusqu'ici, les Championnats ont toujours été gratuits et nous estimons souhaitable que cela reste ainsi. Dans la mesure où les subventions locales, les sponsors ou encore certaines aides de l'AFS permettent aux équipes organisatrices d'engager peu voire pas de dépenses, il nous semble donc qu'une participation payante à la compétition soit facilement évitable. Cependant, rien ne vous empêche de proposer une inscription payante si vous le souhaitez vraiment et l'estimez nécessaire et/ou justifiée, tout en gardant bien à l'esprit que la Commission se réserve le droit de juger de la pertinence de cette requête.


\subsection*{Prix et récompenses}
La remise d'un certificat et d'une coupe aux trois premiers de chaque épreuve est obligatoire. Les coupes peuvent éventuellement être remplacées par des médailles.

Plusieurs types de prix et récompenses sont possibles. Les éditions précédentes ont notamment proposé des prix sous forme d'argent, des lots pour les podiums, des invitations aux championnats internationaux, des remboursements du trajet aux Championnats de France pour les \emph{subX} ou les \emph{N} premiers.

Afin de rendre les Championnats de France attractifs et accessibles pour le plus grand nombre, il sera apprécié qu'une partie du budget "lots" soit consacrée au remboursement des trajets, dans les limites de votre choix.


\subsection*{Dispositions supplémentaires}

Aucune disposition supplémentaire n'est requise, cependant un partenariat pour le logement, le transport ou la restauration par exemple, sera un plus.
Bien entendu, la réalisation d'idées nouvelles, originales, ou attendues par la communauté, sera sûrement appréciée !


\section*{Décision de la Commission}

L'équipe sélectionnée par la Commission pour organiser les Championnats de France 2019 en sera notifiée le 25 mars 2018.

La décision sera rendue publique lors des Championnats de France 2017 (Lyon, 31 mars - 2 avril). L'annonce officielle sera faite conjointement par un membre de l'équipe organisatrice et par un membre de la Commission. Une fois l'annonce officielle passée, il sera à la charge de l'équipe organisatrice d'annoncer l'événement par les voies de communication de son choix (forums, réseaux sociaux, …) dans un délai d'une semaine.

Suite à cette annonce, la Commission et l'équipe organisatrice travailleront conjointement à un "suivi de l'organisation", tel que décrit dans la Charte de la Commission.


\section*{Informations complémentaires}

Il nous semble important que vous sachiez à qui vous vous adressez ; la Commission est composée des 4 membres décisionnaires suivants :

\begin{itemize}
    \item Pauline Bonnaudet (\href{https://www.worldcubeassociation.org/results/p.php?i=2009BONN01}{2009BONN01})
    \item Emilien Fabre (\href{https://www.worldcubeassociation.org/results/p.php?i=2012FABR01}{2012FABR01})
       \item Arthur Garcin (\href{https://www.worldcubeassociation.org/results/p.php?i=2014GARC27}{2014GARC27})
    \item Hippolyte Moreau (\href{https://www.worldcubeassociation.org/results/p.php?i=2008MORE02}{2008MORE02})
\end{itemize}

\paragraph{}
Antoine Piau (\href{https://www.worldcubeassociation.org/results/p.php?i=2008PIAU01}{2008PIAU01}) apporte également une aide technique mais n'a aucun rôle décisionnel.

\paragraph{}
Nous avons tous contribué à organiser et participé à de multiples compétitions, et nous utiliserons notre expérience pour choisir le meilleur candidat parmi les dossiers soumis. De plus, au-delà du choix du dossier retenu pour l'édition 2019 des championnats de France, nous sommes également là pour vous conseiller si vous en avez besoin. N'hésitez donc surtout pas à prendre contact avec nous à l'adresse \href{mailto:cdf-2019@speedcubingfrance.org}{cdf-2019@speedcubingfrance.org}.

\pagebreak
\newcommand{\chartefile}{../Charte/Charte_Commission.pdf}
\IfFileExists{\chartefile}{ \includepdf[pages=-]{\chartefile} } {}

%=============	
\end{document}
%=============