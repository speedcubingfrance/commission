\documentclass[a4paper,12pt]{scrartcl}
\usepackage[usenames,dvipsnames]{color}

\usepackage{pgfplots}
\usepackage{fancyhdr}
\usepackage[utf8]{inputenc}				% encodage UTF-8
\usepackage[T1]{fontenc}
\usepackage[francais]{babel}
\usepackage{sectsty}

\usepackage[colorlinks=true, urlcolor=MidnightBlue, linkcolor=Black]{hyperref}

\pagestyle{fancy}
\setkomafont{disposition}{\normalfont\bfseries}

\newcommand{\documenttitle}{Charte de l'AFS pour la Commission pour l'attribution des Championnats de France}

\lfoot{}
\cfoot{\footnotesize \documenttitle}
\rfoot{\thepage}
\renewcommand{\headrulewidth}{0.0pt}
\renewcommand{\footrulewidth}{0.4pt}
\renewcommand{\sectionmark}[1]{ \markright{#1}{} }

\fancyhead[R]{}
\fancyhead[CL]{}

\textheight 24cm
\hoffset 0mm
\voffset 0.5cm

\sectionfont{
    \sectionrule{0pt}{0pt}{-5pt}{0.8pt}
}



%===============
\begin{document}
%===============

\begin{titlepage}
    \centering
    \vfill
    \includegraphics[width=\textwidth]{logoafsletters.png}
    \vfill
    {\bfseries\Huge
        \documenttitle \\
        \vskip5cm
\today
    }    
    \vfill
\end{titlepage}


\pagebreak

%---------
%	Introduction
%---------

\section*{Choix des membres de la Commission}
La Commission est choisie durant l'Assemblée Générale de l'AFS. Les candidats qui ne pourraient y être présent peuvent se faire représenter par une tierce personne munie de leur candidature écrite et signée. Les membres sont élus par vote de tous les adhérents présents à l'AG.

La Commission comprend cinq personnes. Pour être candidat, il est demandé d'avoir participé à au moins 8 compétitions officielles de la World Cube Association. Il est également exigé d'avoir déjà fait partie d'une équipe organisatrice de compétition afin de mieux comprendre les besoins d'une organisation. La présence d'un délégué dans la Commission peut être souhaitable mais n'est pas nécessaire.

La présence au sein de la Commission est limitée à deux années consécutives, afin de favoriser l'implication de nombreux membres de la communauté.

En cas d'absence de suffisamment de candidats répondant aux critères d'éligibilité, la Commission peut être réduite jusqu'à trois membres. Les membres de la Commission s'engagent à être impartiaux dans leur choix, à n'être impliqué dans aucun projet candidat, et à remplir leur mission avec attention tout au long de l'année.

\section*{Les missions de la Commission}
La Commission a deux missions :

\subsection*{1) Procéder au choix de l'équipe d'organisation}
Chaque année, la Commission publie un appel à candidatures pour l'organisation du Championnat de France, lequel appel sera publié sur le site de l'AFS, ainsi que sur les supports extérieurs (forums, autres sites, etc.) choisis par la Commission. Avec cet appel d'offre, la Commission édite chaque année un document expliquant ses attentes exactes. L'historique des documents réalisés par les diverses Commissions est disponible sur le \href{http://www.speedcubingfrance.org/speedcubing/cdf_historique/}{site de l'AFS}.

Pourra répondre à cet appel d'offre toute équipe d'organisation qui le souhaitera. Les équipes candidates devront envoyer à la Commission un dossier présentant le détail de leur candidature. Une date limite d'envoi sera définie. Le choix de cette date sera effectué à la discrétion de la Commission. À l'issue de cette période de candidature, les membres de la Commission devront choisir parmi les dossiers candidats et selon leurs propres critères celui qui leur paraît le plus à même d'être bénéfique pour la communauté française du speedcubing. Ses choix se baseront notamment sur l'accessibilité de la ville, l'alternance géographique du Championnat, les capacités d'organisation déjà démontrées des équipes candidates ainsi que sur la qualité des infrastructures qui lui sont proposées. Elle doit également demander des détails techniques concrets comme un planning. La Commission s'engage à ne pas divulguer le contenu des dossiers qui lui seront envoyés. Elle annoncera son choix dès que possible. Elle privilégiera dans la mesure du possible l'annonce chaque année au cours du Championnat de France de l'équipe et de la ville qui prendront en charge le prochain.

Si aucun dossier déposé n'apporte satisfaction, la Commission se réserve le droit de prolonger le délai pour les candidatures, voire de prendre elle-même en charge l'organisation en cas d'urgence. En dehors de ce cas de figure, ses membres s'engagent à ne faire partie d'aucune équipe d'organisation qui proposerait un dossier.

\subsection*{2) Procéder à un suivi de l'organisation du Championnat}
Une fois le choix effectué et annoncé, il sera du ressort de la Commission de vérifier que l'organisation du Championnat avance de manière régulière. Elle pourra imposer à l'organisateur des dates limites, par exemple pour effectuer publiquement certaines annonces, comme l'adresse exacte, les horaires, le planning, ou pour effectuer certaines actions comme ouvrir les inscriptions ou mettre en ligne un site web.

Tout ceci s'effectuera dans le dialogue avec l'équipe d'organisation, dans le but commun de réaliser le meilleur Championnat possible pour la communauté. La Commission, composée d'organisateurs et de compétiteurs expérimentés, joue le rôle de consultant, et en cas de difficulté, doit être disponible pour aider et conseiller l'équipe d'organisation.

Si néanmoins l'équipe d'organisation venait à échouer à remplir ses engagements, la Commission, en dernier recours et après plusieurs avertissements, pourrait lui retirer le privilège de l'organisation du Championnat, et le transférer à une autre équipe candidate si possible, sinon l'organiser elle-même.

\section*{Les engagements des équipes candidates}
Les équipes d'organisation candidates, en remettant un dossier à la Commission, s'engagent à :
\begin{itemize}
\item Dès le premier dossier, respecter les contraintes données par la Commission dans son document relatif aux candidatures. Tout dossier s'éloignant manifestement des demandes de la Commission sera directement écarté.
\item Avoir pour objectif de réaliser une compétition à la hauteur d'un événement tel que le Championnat de France, et d'organiser la compétition au bénéfice de la communauté du speedcubing.
\item Proposer un Championnat ouvert à tous et incluant toutes les épreuves officielles de la World Cube Association.
\item Être réactives aux demandes de la Commission, y répondre dès réception et les satisfaire dans les délais.
\item En cas de soucis ou d'imprévu, en avertir directement la Commission afin que des solutions soient rapidement trouvées.
\end{itemize}


%=============	
\end{document}
%=============

 
